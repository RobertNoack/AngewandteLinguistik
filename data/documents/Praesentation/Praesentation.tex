\documentclass[12pt]{beamer}
\usetheme{Frankfurt}
\usecolortheme{seahorse}
\usepackage[utf8]{inputenc}
\usepackage[german]{babel}
\usepackage[T1]{fontenc}
\usepackage{amsmath}
\usepackage{amsfonts}
\usepackage{amssymb}
\usepackage{graphicx}
\author{Robert Noack, Stefan Schaub, Ramon Bernert}
\title{Themenbereich 8: \\Daten des Universitätsarchivs}

%\setbeamercovered{transparent} 
%\setbeamertemplate{navigation symbols}{} 
%\logo{} 
%\institute{} 
\date{6. Juli 2015} 
%\subject{} 
\begin{document}

\section*{}
\begin{frame}
\titlepage
\end{frame}

\begin{frame}
\tableofcontents
\end{frame}

\section{Einführung}
 \subsection*{~}
\begin{frame}{Einleitung}

\end{frame}

\begin{frame}

\end{frame}

\section{Ausgangsdaten}
 \subsection*{~}
\begin{frame}{Datenherkunft}

\end{frame}

\begin{frame}{Struktur der Ausgangsdaten}

\end{frame} 

\begin{frame}{Gewünschte Struktur}

\end{frame}


\section{Quelldatenfehler}
 \subsection*{~}
\begin{frame}{Fehler in den Ausgangsdaten}

\end{frame}

\section{Verwendete Verfahren}
 \subsection*{~}
\begin{frame}{Python Code}

\end{frame}

\begin{frame}{Geotag}

\end{frame}

\section*{}
\begin{frame}
\centering
{\Large Vielen Dank für Ihre Aufmerksamkeit. \\ Noch Fragen?}
\end{frame}


\end{document}
