\documentclass[12pt]{beamer}
\usetheme{Frankfurt}
%\usecolortheme{seahorse}
\usepackage[utf8]{inputenc}
\usepackage[german]{babel}
\usepackage[T1]{fontenc}
\usepackage{amsmath}
\usepackage{amsfonts}
\usepackage{amssymb}
\usepackage{graphicx}


\author{Robert Noack, Stefan Schaub, Ramon Bernert}
\title{\textbf{Themenbereich 8: \\Daten des Universitätsarchivs}}

%\setbeamercovered{transparent} 
\setbeamertemplate{navigation symbols}{} 
%\logo{} 
%\institute{} 
\date{6. Juli 2015} 
%\subject{} 


\begin{document}
\begin{large}
\section*{}
\begin{frame}
\titlepage
\end{frame}

\begin{frame}
\tableofcontents
\end{frame}

\section{Einführung}
 \subsection*{~}
\begin{frame}{Einleitung}

\end{frame}

\begin{frame}

\end{frame}

\section{Ausgangsdaten}
 \subsection*{~}
\begin{frame}{Datenherkunft}

\end{frame}

\begin{frame}{Struktur der Ausgangsdaten}
\begin{itemize}
\item 
\item 3 Infromationsbereiche
 \begin{enumerate}
  \item Namensbereich
  \item persönliche Informationen
  \item Dokumente
 \end{enumerate}
\end{itemize}

 
 \vspace*{\fill}
 \begin{block}{Beispiel}
 \normalsize Namensbereich: persönliche Informationen-{}- Dokumente.-
 \end{block}
\end{frame} 

\begin{frame}{Namensbereich}
 Nachname, Vorname$_1$ Vorname$_2$ ... Vorname$_n$:

 \vspace*{\fill}
 \begin{block}{Beispiel}
 \normalsize Ahnemüller, Gottlob Wilhelm:
 \end{block}
\end{frame}


\begin{frame}{persönliche Informationen}
 akademischer Titel, Geburtsort Geburtsjahr--

 \vspace*{\fill}
 \begin{block}{Beispiele}
  \normalsize
  \begin{itemize}
   \item stud. oecon., geb. Kiel 5.9.1897-{}-
   \item test
  \end{itemize}
 \end{block}
\end{frame}


\begin{frame}{Dokumente}



 \vspace*{\fill}
 \begin{block}{Beispiel}
 \normalsize text hier einfügen
 \end{block}
\end{frame}


\begin{frame}{Gewünschte Struktur}



\end{frame}


\section{Quelldatenfehler}
 \subsection*{~}
\begin{frame}{Fehler in den Ausgangsdaten}
 \begin{itemize}
  \item bsp1
  
 \end{itemize}
\end{frame}

\section{Verwendete Verfahren}
 \subsection*{~}
\begin{frame}{Ablauf}
 \begin{itemize}
  \item regex Filter für die 3 Bereiche
 \end{itemize}
\end{frame}

\begin{frame}{Python Code}

\end{frame}

\begin{frame}{Geotag}

\end{frame}

\begin{frame}{zukünftige Verfahren}
 \begin{itemize}
   \item 
 \end{itemize}
\end{frame}


\section*{}
\begin{frame}
\centering
{\Large Vielen Dank für Ihre Aufmerksamkeit. \\ Noch Fragen?}
\end{frame}

\end{large}
\end{document}
