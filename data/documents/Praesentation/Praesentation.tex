\documentclass[12pt]{beamer}
\usetheme{Frankfurt}
%\usecolortheme{seahorse}
\usepackage[utf8]{inputenc}
\usepackage[german]{babel}
\usepackage[T1]{fontenc}
\usepackage{amsmath}
\usepackage{amsfonts}
\usepackage{amssymb}
\usepackage{graphicx}
\usepackage{xcolor}

\author{Robert Noack, Stefan Schaub, Ramon Bernert}
\title{\textbf{Themenbereich 8: \\Daten des Universitätsarchivs}}

%\setbeamercovered{transparent} 
\setbeamertemplate{navigation symbols}{} 
%\logo{} 
%\institute{} 
\date{6. Juli 2015} 
%\subject{} 


\begin{document}
\begin{large}
\section*{}
\begin{frame}
\titlepage
\end{frame}

\begin{frame}
\tableofcontents
\end{frame}

\section{Einleitung}
 \subsection*{~}
\begin{frame}{Datenherkunft}
 \begin{itemize}
  \item Daten über zurückgelassene Dokumente von Studenten
  \item stammen aus dem Universitätsarchiv
  \item Reichen bis ins 19. Jhr. zurück
  \item liegen in xls-Format vor
  \item Tabelle mit 41 Zeilen
  \item Informationen zu mehreren Studenten in einer Zelle
 \end{itemize}

\end{frame}


\begin{frame}{Vorgehen}
 \begin{itemize}
  \item Zellen aus xls-Datei in Textdatei kopiert
  \item Pyhton-Script analysiert diese Datei
  \item Ordnet zunächst die Daten anhand einfacher Muster den einzelnen Studenten zu
  \item Danach detaillierte Analyse der Zeichenkette pro Student
  \item Ausgabeformat: JSON
 \end{itemize}
\end{frame}


\begin{frame}{Struktur der Ausgangsdaten}
3 Informationsbereiche
 \begin{enumerate}
  \item Namensbereich
  \item persönliche Informationen
  \item Dokumente
 \end{enumerate}

 \vspace*{\fill}
 \begin{block}{Beispiel}
 \normalsize Namensbereich: persönliche Informationen-{}- Dokumente.-
 \end{block}
\end{frame} 


\section{Strukturierung}
 \subsection*{~}
 
\begin{frame}{Namensbereich}
 Nachname, Vorname$_1$ Vorname$_2$ ... Vorname$_n$:

 \vspace*{\fill}
 \begin{block}{Beispiel}
 \normalsize Ahnemüller, Gottlob Wilhelm:
 \end{block}

 \begin{block}{Struktur}
  \normalsize
  \{
  \newline
  \hspace*{0.5cm}
  "{}surname"{}: "{}Ahnemüller"{},
  \newline
  \hspace*{0.5cm}    
  "{}prename"{}: "{}Gottlob Wilhelm"{},
 \end{block} 
\end{frame}


\begin{frame}{persönliche Informationen}
 : akademischer Titel, Geburtsort Geburtsjahr-{}-

 \vspace*{\fill}
 \begin{block}{Beispiele}
  \normalsize
  \begin{itemize}
   \item stud. oecon., geb. Kiel 5.9.1897-{}-
   \item geb. 5.7.1860 in  Zöblitz-{}-
   \item aus Saalfeld-{}-
  \end{itemize}
 \end{block}
 
 \begin{block}{Struktur}
  \normalsize
  \hspace*{0.5cm}
  "{}adcademic title"{}: "{}stud. oecon."{},
  \newline 
  \hspace*{0.5cm}    
  "{}birthplace"{}: "{}Kiel"{},
  \newline
  \hspace*{0.5cm}
  "{}birthdate"{}: "{}5.9.1897"{},    
  \newline
  \hspace*{0.5cm}
  "{}additional\_information\_of\_birthplace"{}: "{}null"{},
 \end{block} 
\end{frame}

\begin{frame}{Dokumente}
 -{}- Dokument$_1$, Dokument$_2$, ..., Dokument$_n$.-

 \vspace*{\fill}
 \begin{block}{Beispiel}
  \normalsize
  \begin{itemize}
   \item Entlastungsschein Universitätsbibliothek 1910, Abgangszeugnis Universität Leipzig 1910.-
   \item Abgangszeugnis Universitätsgericht Leipzig 1842.-
   \item G, M, väterliche/vormundschaftliche Studiengenehmigung 1872.-
  \end{itemize}
 \end{block}
\end{frame}

\begin{frame}{Dokumenttypen}
 \begin{itemize}
  \item Zeugnis
  \item Buch
  \item Schein
  \item Protokoll
  \item Genehmigung
  \item Zuweisung
  \item Diplom
  \item Quittung
 \end{itemize}
\end{frame}

\begin{frame}{Dokumente}
 \begin{itemize}
  \item statische Methode Certificate.getCertificates(value) erzeugt aus String entsprechende Certificate-Instanzen
  \item Zerlegt String anhand von Komma
  \item Regex .*(\textbackslash d\{4\}([/]\textbackslash d\{2\})*).* sucht nach Jahreszahl
  \item Jedes Wort wird an Location.getLocation(city) übergeben um Ort zu finden
  \item Location.getYear() gibt Jahreszahl zurück
 \end{itemize}
\end{frame}

\begin{frame}{Dokumente}
 \begin{block}{Struktur}
  \normalsize
  \hspace*{0.5cm} 
  "'certificate"{}: [\{
\newline 
\hspace*{1cm}"{}source"{}: "{}Reifezeugnis 1866"{},
\newline 
\hspace*{1cm}"{}name"{}: "{}Reifezeugnis"{},
\newline 
\hspace*{1cm}"{}type"{}: "{}Zeugnis"{},
\newline 
\hspace*{1cm}"{}year"{}: "{}1866"{}
\newline 
\hspace*{0.5cm}\},\{
\newline 
\hspace*{1cm}"{}source"{}: "{}Abgangszeugnis Universität Leipzig 1873"{},
\newline 
\hspace*{1cm}"{}name"{}: "{}Abgangszeugnis Universität Leipzig"{},
\newline 
\hspace*{1cm}"{}location"{}: \{
\newline 
\hspace*{1cm}\hspace*{0.5cm}"{}name"{}: "{}Leipzig"{},
\newline 
\hspace*{1cm}\},
\newline 
\hspace*{1cm}"{}type"{}: "{}Zeugnis"{},
\newline 
\hspace*{1cm}"{}year"{}: "{}1873"{}\}]
 \end{block} 
\end{frame}


\begin{frame}{Gewünschte Struktur}
 \begin{block}{Beispiel}
  \small
  Abb, Edmund: geb. Trennfurt, Bez. Obernberg 8.6.1878-{}- Prüfungszeugnis 1902.-
 \end{block}
\end{frame}

\begin{frame}
 \begin{block}{Struktur}
  \small
  \{
  \newline
  \hspace*{0.5cm}
  \textcolor{green}{"{}surname"{}: "{}Abb"{},}
  \newline
  \hspace*{0.5cm}
  \textcolor{green}{"{}prename"{}: "{}Edmund"{},}
  \newline 
  \hspace*{0.5cm}
  \textcolor{red}{"{}adcademic title"{}: "{}null"{},}
  \newline 
  \hspace*{0.5cm}
  \textcolor{red}{"{}birthplace"{}: "{}Trennfurt"{},}
  \newline
  \hspace*{0.5cm}
  \textcolor{red}{"{}birthdate"{}: "{}8.6.1878"{},}
  \newline
  \hspace*{0.5cm}
  \textcolor{red}{"{}additional\_information\_of\_birthplace"{}: "{}Bez. Obernberg"{},}
  \newline
  \hspace*{0.5cm}
  \textcolor{cyan}{"{}certificate"{}: [$\{$
%  "{}Prüfungszeugnis 1902"{},}
%  "{}certificate"{}: 
  \newline 
  \hspace*{1cm}"{}source"{}: "{}Prüfungszeugnis 1902"{},
  \newline 
  \hspace*{1cm}"{}name"{}: "{}Prüfungszeugnis"{},
  \newline 
  \hspace*{1cm}"{}type"{}: "{}Zeugnis"{},
  \newline 
  \hspace*{1cm}"{}year"{}: "{}1902"{}$\}$ ]} 
  \newline
  \hspace*{0.5cm}
  "{}object\_under\_investigation"{}: "{}Abb, Edmund: geb. Trennfurt,\\
  \hspace*{0.5cm}
   Bez. Obernberg 8.6.1878-{}- Prüfungszeugnis 1902"{}
  \newline
  \}
 \end{block} 
\end{frame}


\section{Quelldatenfehler}
 \subsection*{~}
\begin{frame}{Fehler in den Ausgangsdaten}
 \begin{itemize}
  \item bsp1
 \end{itemize}
\end{frame}

\section{Erweiterte Struktur}
 \subsection*{~}

\begin{frame}{Geotagging}
\begin{itemize}
  \item Zur Identifizierung von Orten: geonames.org
  \item Bietet einfache API
  \item gibt JSON zurück
  \item enthält geonameID, Längen- und Breitengrad
  \item Aufruf des Webservices erfolgt in Klasse Location
  \item Verwendung zum Parsen des Geburts- und Dokumentenorts
 \end{itemize}
\end{frame}
 
\begin{frame}{Geotagging}
 \vspace*{\fill}
 \begin{block}{Aufruf}
 \normalsize http://api.geonames.org/searchJSON?q=Leipzig\&username=...
 \end{block}
 
 \begin{block}{Ausgabe}
 \normalsize \{"{}totalResultsCount"{}:208,
  \newline
  \hspace*{0.5cm}"{}geonames"{}:[\{
  \newline
  \hspace*{1cm}"{}lng"{}:"{}12.37129"{},
  \newline
  \hspace*{1cm}"{}name"{}:"{}Leipzig"{},
  \newline  
  \hspace*{1cm}"{}geonameId"{}:2879139,
  \newline
  \hspace*{1cm}"{}lat"{}:"{}51.33962"{}
  \newline
  \hspace*{0.5cm}\}]
  \newline
  \}
 \end{block}
\end{frame} 
 
\begin{frame}{Geotagging}
 \begin{itemize}
  \item statische Methode Location.getLocation(city) erzeugt aus String eine Location-Instanz
  \item Instanzen werden gecached
  \item Location.getLng() gibt Längengrad zurück
 \end{itemize}
 
% \vspace*{\fill}
 \begin{block}{Erzeugtes JSON}
 \normalsize \{
 \newline
 \hspace*{0.5cm}"{}name"{}: "{}Leipzig"{},
 \newline
 \hspace*{0.5cm}"{}latitude"{}: "{}51.33962"{},
 \newline
 \hspace*{0.5cm}"{}geonameId"{}: 2879139,
 \newline
 \hspace*{0.5cm}"{}longitude"{}: "{}12.37129"{},
 \newline
 \hspace*{0.5cm}"{}url"{}: 
 \hspace*{0.5cm}"{}https://maps.google.de/maps?q=51.33962,12.37129"{}
 \newline
 \}
 \end{block}
\end{frame}
 

\begin{frame}{zukünftige Verfahren}
 \begin{itemize}
   \item 
 \end{itemize}
\end{frame}


\section*{}
\begin{frame}
\centering
{\Large Vielen Dank für Ihre Aufmerksamkeit. \\ Noch Fragen?}
\end{frame}

\end{large}
\end{document}
