\subsection{Ortsbestimmung}
 \label{subsec:Ortsbestimmung}
 Die vorliegenden Personendaten enthalten zahlreiche Ortsangaben, z.B. den Geburtsort oder den Ausstellungsort der Dokumente. Um die Korrektheit dieser Ortsnamen zu \"{u}berpr\"{u}fen, verwendet die API den Webservice "{}geonames.org"{}. Der Webservice kann nach einer Registrierung kostenlos verwendet werden. Als Eingabe erwartet dieser einen Orts- und den Nutzernamen. Die Ausgabe besteht aus einer Auflistung aller gefundenen Orte mit zahlreichen Eigenschaften, wie das Land, oder den L\"{a}ngen- und Breitengrad. Der Webservice wird \"{u}ber einen HTTP-GET-Request nach folgendem Schema angesprochen:
\begin{itemize}
\item http://api.geonames.org/searchJSON\newline ?q=\{Ortsname\}\& username=\{Nutzer\}
\end{itemize}
Die Ausgabe erfolgt im JSON-Format, wobei jeder Ort ein eigenes Objekt mit den dazugeh\"{o}rigen Eigenschaften darstellt. Die folgende Anfrage 
\begin{itemize}
\item http://api.geonames.org/searchJSON\newline ?q=leipzig\& username=asv2015
\end{itemize}
liefert beispielsweise diese Ausgabe:
\begin{lstlisting}[language=JSON]
{
  "totalResultsCount":941,
  "geonames":[
  {
    "countryId":"2921044",
    "countryName":"Germany",
    "countryCode":"DE",
    "lng":"12.37129",
    "name":"Leipzig",
    "geonameId":2879139,
    "lat":"51.33962"
  },
  ...]
}
\end{lstlisting}
Dabei stellt jedes Element der Eigenschaft "{}geonames"{} einen zu der Suchanfrage passenden Ort dar. Da diese Auflistung nach der Relevanz sortiert ist, verwendet unsere API nur den ersten Eintrag und behandelt diesen als korrektes Ergebnis. Um den Webservice innerhalb der API einfach verwenden zu k\"{o}nnen, implementiert diese die Klasse "{}Location"{}. Diese Klasse besitzt die Methoden "{}getGeonameId"{}, "{}getName"{}, "{}getLng"{}, "{}getLat"{} und "{}getUrl"{}. Diese Methoden repr\"{a}sentieren die entsprechenden Werte des Ergebnisses einer Suchanfrage. Dabei gibt "{}getUrl"{} die Google-maps-Url zu dem Ort zur\"uck. Zus\"atzlich enth\"alt die Klasse noch die statische Methode "{}getLocation"{}, welche als Parameter den zu suchenden Ortsnamen erwartet und eine Instanz von "{}Location"{} zur\"uck gibt. Die Methode erstellt anhand der Webservice-Url und dem \"ubergebenen Ortsnamen einen HTTP-Request, welcher an den Webservice gesendet wird. Aus der daraus resultierenden Antwort wird ein JSON-Objekt erzeugt und die enthaltenden Werte auf die Eigenschaften einer neuen "{}Location"{}-Instanz \"ubertragen. Anschlie{\ss}end wird diese Instanz zur\"uck gegeben. Ein gro{\ss}er Nachteil des Webservices ist, dass dieser Zugriffsbeschr\"{a}nkt ist. Das hei{\ss}t, innerhalb einer Stunde d\"urfen nur 2000 Anfragen get\"{a}tigt werden und diese Anzahl reicht nicht aus um jeden Ortsnamen erneut bei dem Webservice abzurufen. Um diese Einschr\"{a}nkung zu umgehen, besitzt die "{}Location"{} ein statisches "{}Dictionary"{}, welches zu jedem \"uber die Methode "{}getLocation"{} angefragten Ort die dazugeh\"orige "{}Location"{}-Instanz speichert. Somit schaut "{}getLocation"{} vor dem Absenden des Requests in diesem "{}Dictionary"{} ob der angefragte Ort schon vorhanden ist. Wenn das der Fall ist, wird diese gecachte Instanz zur\"uck gegeben.

