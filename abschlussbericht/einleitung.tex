\section{Einleitung}
 \label{sec:Einleitung}

In dieser Seminararbeit geht es um die Aufarbeitung von halbstrukturierten Daten. Bei den Daten handelt es sich um alte Personendaten aus dem Universitätsarchiv. Diese wurden zwar bereits durch die Verwendung von bestimmten Zeichenketten (beispielsweise .- für den Zeilenumbruch) strukturiert, diese Struktur wurde aber bei einem Bearbeitungsschritt leicht beschädigt und reicht für die Weiterverarbeitung der Daten nicht aus. Aus diesem Grund war es Ziel dieser Arbeit, die Daten aus dem einfachen .txt-Format in ein strukturiertes JSON-File zu überführen. In Kapitel \ref{sec:Datenherkunft und interne Struktur} werden wir genauer auf die Daten und ihre interne Struktur eingehen und vorliegende Probleme aufzeigen. Im Anschluss wird in Kapitel \ref{sec:Verwendete Methoden} die Funktionalität unseres Codes erklärt und auf die zusätzlich verwendete API verwiesen. Zum Abschluss ist in Kapitel \ref{sec:Beispielstruktur} ein Beispiel für unsere komplette JSON-Struktur angegeben.